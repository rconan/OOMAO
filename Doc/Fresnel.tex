% To be compiled with pdflatex
% This file is to be included into master file via \input command
% Note that there is no \begin{document} \end{document} brackets!

\newpage
\section{Appendix: \\ Theory of Generalized Fresnel Transform}
\label{app:Fresnel}

\mbox{}

Generalized Fresnel Transform (GFT) is a Fresnel diffraction integral
generalized to propagation through a paraxial optic system defined through
the \emph{wave}- or \emph{ABCD}-matrix \index{wave matrix} \index{ABCD-matrix}
\cite{SchmidtProp,PalmaBagini}
\begin{equation} \label{eq:ABCD-matrix}
  W =
  \left[
  \begin{array}{cc}
    A & B \\
    C & D \\
  \end{array}
  \right], \,\, |W| = AD-BC = 1.
\end{equation}
The GFT is defined as
\begin{equation} \label{eq:GFT-definition}
  U(\bm{y}) = \mathbb{F}[A,B,D] \{ U(\bm{x}) \} (\bm{y}) =
\end{equation}
$$
  \frac{e^{ikz}}{i \lambda B} \int_{-\infty}^{\infty} U(\bm{x})
  \exp \left[
  i \frac{k}{2B} (A|\bm{x}|^{2} - 2 \bm{x}^{T}\bm{y} + D |\bm{y}|^{2})
  \right] d\bm{x}, \,\, k = \frac{2 \pi}{\lambda},
$$
where $\bm{x,y}$ are the xy-coordinates in the source and observation planes,
respectively, $z$ is the distance between the planes, $U$ is a complex
amplitude of the scalar field at the planes. Obviously, when
considering free-space propagation, i.e. when
\begin{equation} \label{eq:free-space-ABCD}
  W =
  \left[
  \begin{array}{cc}
    1 & z \\
    0 & 1 \\
  \end{array}
  \right],
\end{equation}
Eq. (\ref{eq:GFT-definition}) turns into classical Fresnel diffraction
integral. Ref. \cite{PalmaBagini} develops theory of the GFT in the form of
several theorems. We give here a subset of the most useful GFT theorems
translated from 1D to 2D case.
\\ \\
\textbf{Cast-to-convolution theorem.} \\
At first glance Eq. (\ref{eq:GFT-definition}) does not look like a convolution
but it is possible to reduce it to convolution after change of variables:
\begin{equation} \label{eq:convolution-cast}
  U(\bm{y}) = \frac{e^{ikz}}{i \lambda B}
  \exp \left( i\frac{kC}{2A}|\bm{y}|^{2} \right)
  \int_{-\infty}^{\infty} U(\bm{x})
  \exp \left( i \frac{kA}{2B} |\frac{\bm{y}}{A}-\bm{x}|^{2} \right) d\bm{x}.
\end{equation}
This theorem allows to use Fast Fourier Transform for computing GFTs.
The convolution kernel in Eq. (\ref{eq:convolution-cast}), the \emph{GFT chirp},
\index{GFT-chirp} has analytical Fourier Transform:
\begin{equation} \label{eq:GFT-chirp-FT}
  \mathcal{F}
  \{
  \exp \left( i \frac{kA}{2B} |\bm{x}|^{2} \right)
  \} (\bm{f}) =
  \frac{i}{\alpha} \exp \left[ -i \frac{\pi}{\alpha} |\bm{f}|^{2} \right],
  \,\, \alpha = \frac{A}{\lambda B}.
\end{equation}

Eq. (\ref{eq:convolution-cast}) has simple interpretation: propagation through
optical system described by an ABCD-matrix consists of 3 steps
\begin{enumerate}
  \item Free space propagation by distance $B/A$ (convolution term in Eq.
  (\ref{eq:convolution-cast})).
  \item Scaling the propagation result by magnification $A$. If, for instance,
  the first step is done through FFT on a discrete grid with spacing $dx$ in
  the source plane, then in the observation plane the grid spacing is equal to
  $Adx$.
  \item Multiplication by a phase mask corresponding to negative thin lens with
  focal distance $f = -A/C$. \\
\end{enumerate}
\textbf{Shift theorem.}
\begin{equation} \label{eq:shift-theorem}
  \mathbb{F}[A,B,D] \{ U(\bm{x}-\bm{x}_{0}) \} (\bm{y}) =
  \exp \left[
  i \frac{k}{2} C \bm{x}_{0}^{T} (2\bm{y}-A\bm{x}_{0})
  \right]
  \mathbb{F}[A,B,D] \{ U(\bm{x}) \} (\bm{y}-A\bm{x}_{0}).
\end{equation}
\\ \\
\textbf{Tilt theorem.}
\begin{equation} \label{eq:tilt-theorem}
  \mathbb{F}[A,B,D] \{ U(\bm{x}) \exp [ik\bm{x}_{0}^{T}\bm{x}] \} (\bm{y}) =
  \exp \left[
  ik D \bm{x}_{0}^{T} (\bm{y}-\frac{B}{2}\bm{x}_{0})
  \right]
  \mathbb{F}[A,B,D] \{ U(\bm{x}) \} (\bm{y}-B\bm{x}_{0}).
\end{equation}
\\ \\
\textbf{Focus theorem.}
\begin{equation} \label{eq:focus-theorem}
  \mathbb{F}[A,B,D] \{ U(\bm{x})
  \exp \left[ i \frac{k}{2f} |\bm{x}|^{2} \right] \}
  (\bm{y}) =
  \mathbb{F}[A+\frac{B}{f},B,D] \{ U(\bm{x}) \} (\bm{y}).
\end{equation}
Note that, to keep the wave matrix unimodular after $A^{\prime} = A +
\frac{B}{f}$ transformation, the $C$-coeffficient has to be transformed into
\begin{equation} \label{eq:c-correction}
  C^{\prime} = C + \frac{D}{f}.
\end{equation}
Shift, tilt and focus theorems allow to factor possibly highly oscillating
terms out of the integral. To see this for focus theorem, Eq.
(\ref{eq:focus-theorem}) needs to be substituted into Eq.
(\ref{eq:convolution-cast}).